%        File: appunti.interazioni.fondamentali.tex
%     Created: lun nov 21 03:00  2011 C
% Last Change: lun nov 21 03:00  2011 C
%
\documentclass[italian,a4paper]{article}
\usepackage{babel}
\usepackage{feynmf}
\usepackage{amssymb,amsmath,amsthm}
\usepackage{mathrsfs}
\usepackage{nicefrac}
\usepackage[text={6in,9in},centering]{geometry}
\usepackage[utf8x]{inputenc}
\usepackage[T1]{fontenc}
\usepackage{ae,aecompl}
\usepackage[bf,footnotesize]{caption}
\frenchspacing
\pagestyle{plain}

\theoremstyle{definition}
\newtheorem{definition}{Definizione}[section]

\DeclareMathOperator{\tr}{tr}
\newcommand{\lagr}{\ensuremath{\mathscr{L}}}
\newcommand{\dimu}{\ensuremath{\partial_{\mu}}}
\newcommand{\Dimu}{\ensuremath{D_{\mu}}}
\newcommand{\dinu}{\ensuremath{\partial_{\nu}}}
\newcommand{\Dinu}{\ensuremath{D_{\nu}}}
\renewcommand{\leq}{\leqslant}

\title{Teoria delle interazioni fondamentali}
\author{Matteo Abis}
\date{\today}
\begin{document}
\maketitle

\section{Teorie di gauge}
\begin{definition}[Teoria di gauge] una teoria quantistica di campi
    invariante sotto trasformazioni locali di un gruppo di Lie, detto gruppo
    \emph{di gauge}. Le trasformazioni sono locali se i parametri dipendono
    dal punto dello spaziotempo.
\end{definition}

\subsection{Prototipo: l'elettrodinamica quantistica}
\begin{description}
    \item[campi:] un campo spinoriale $\psi(x)$, un campo vettoriale
        $A^{\mu}(x)$;
    \item[trasformazioni di gauge:] sotto l'azione degli elementi del gruppo
        di gauge $U(1)$ i campi trasformano come
        \begin{align*}
            \psi'(x) &= e^{-ie\alpha(x)}\psi(x)\\
            A^{\mu'}(x) &= A^{\mu}(x) + \dimu \alpha(x);
        \end{align*}
    \item[rinormalizzabilit\`a:] compaiono nella lagrangiana soltanto
        termini con dimensione $d \leq 4$.
\end{description}

La lagrangiana pi\`u generale compatibile con questi requisiti \`e dunque:
\begin{equation*}
    \mathscr{L_{\text{QED}}} = -\dfrac{1}{4}F_{\mu\nu}F^{\mu\nu} + i
    \bar{\psi}\gamma^{\mu}(\dimu + i e A_{\mu})\psi - m
    \bar{\psi}\psi
\end{equation*}

\subsection{Teorie di gauge non abeliane}
Discutiamo nel dettaglio la costruzione di una generica teoria di gauge,
seguendo gli stessi passi che ci hanno portato alla formulazione
dell'elettrodinamica quantistica. \`E necessario innanzitutto identificare i
componenti fondamentali della teoria.

\begin{description}
    \item[gruppo di gauge $G$:] deve essere un
        \begin{itemize}
            \item gruppo di Lie. Sia $n$ la sua dimensione;
            \item compatto, perch\'e le rappresentazioni siano unitarie;
            \item semplice, ovvero senza sottogruppi invarianti non banali.
                Questa richiesta non \`e fondamentale e sar\`a eliminata in
                seguito.
        \end{itemize}
    \item[campi di spin \nicefrac{1}{2} e spin $0$:] genericamente indicati
        con il multipletto $\varphi$.
    \item[propriet\`a di trasformazione dei campi: ] il multipletto dei
        campi deve trasformare come una rappresentazione $R$ del gruppo
        $G$. Detti $t^{a}_{R}$ ($a = 1, \dots, n$) i generatori del
        gruppo in tale rappresentazione, e $\alpha_a$  i parametri della
        trasformazione
        \begin{equation*}
            \varphi'(x) = \Omega\varphi = e^{-i \alpha_a t^{a}_{R}}\varphi(x).
        \end{equation*}
        \`E talvolta utile considerare trasformazioni infinitesime
        \begin{equation*}
            \delta\varphi =   -i \alpha_a t^{a}_{R} \varphi.
        \end{equation*}
        Introduciamo infine le costanti di struttura dell'algebra di Lie
        $f^{ab}_{c}$
        \begin{equation*}
            [t^{a}, t^{b}] = if^{ab}_{c}t^{c}
        \end{equation*}
\end{description}

Una volta specificati gli ingredienti, la teoria segue immediatamente
dall'applicazione di una procedura quasi meccanica:
\begin{enumerate}
    \item determinazione della lagrangiana $\lagr(\varphi,
        \partial\varphi)$ pi\`u generale invariante per il gruppo $G$ sotto
        trasformazioni globali, ovvero indipendenti dal punto dello
        spaziotempo;
    \item promozione delle trasformazioni globali in trasformazioni locali.
        A questo punto i termini con le derivate non trasformano pi\`u come
        i campi e la lagrangiana non \`e pi\`u invariante:
        \begin{equation}
            (\dimu\varphi)' = (\dimu \Omega)\varphi +
            \Omega(\dimu\varphi)\neq \Omega(\dimu \varphi).
            \label{eq:trasformazione_derivata}
        \end{equation}
        Si introduce dunque una \emph{derivata covariante}, che trasforma
        come i campi, $\Dimu$
        \begin{equation*}
            \Dimu \varphi = (\dimu + i A_{a\mu}t^{a})\varphi
        \end{equation*}
        dove abbiamo introdotto un campo vettoriale reale \emph{di gauge} $A_\mu = i A_{a\mu}t^{a}$,
        che \`e un elemento dell'algebra di Lie del gruppo $G$. Vogliamo
        infatti che questo termine cancelli il primo addendo
        della~\eqref{eq:trasformazione_derivata}, che \`e un elemento
        dell'algebra di Lie.
        Imponendo quindi la legge di trasformazione gi\`a valida per i campi
        \begin{equation*}
            (\Dimu \varphi)' = \Omega \Dimu \varphi\\
        \end{equation*}
        \begin{equation*}
            (\dimu + A'_{\mu})\Omega \varphi = (\dimu \Omega)\varphi +
            \Omega (\dimu \varphi) + A'_\mu\Omega\varphi = \Omega(\dimu
            \varphi) + \Omega A_\mu \varphi\\
        \end{equation*}
        \begin{equation*}
            (A'_\mu\Omega - \Omega A_\mu + \dimu \Omega)\varphi = 0
        \end{equation*}
        otteniamo la legge di trasformazione per i campi di gauge,
        moltiplicando a destra per $\Omega^{-1}$:
        \begin{equation}
            A_{\mu}' = \Omega A_\mu \Omega^{-1} - (\dimu \Omega)
            \Omega^{-1}.
            \label{eq:trasformazione_campi_gauge}
        \end{equation}

        La~\eqref{eq:trasformazione_campi_gauge} si pu\`o capire meglio in
        termini dei campi $A_{a\mu}$ scrivendola per trasformazioni
        infinitesime:
        \begin{align}
            i A'_{a\mu}t^a &= (1 - i \alpha_b t^{b}) A_{c\mu}t^{c}(1 + i
            \alpha_b t^{b}) - [\dimu(1 - i \alpha_a t^{a})(1 + \cdots)]\nonumber\\
            &= i A_{a\mu}t^a + \alpha_a A_{c\mu}[t^b, t^c] + i \dimu \alpha_a
            t^a\nonumber\\
            &= i(A_{a\mu} + \dimu \alpha_a)t^a + i f^{bc}_a t^a\nonumber\\
            A'_{a\mu} &= A_{a\mu} + \dimu \alpha_a + f^{bc}_a \alpha_b
            A_{c\mu}.\label{eq:trasformazione_infinitesima_gauge}
        \end{align}
        Vediamo dunque che, rispetto al caso abeliano dell'elettrodinamica
        quantistica, si introduce un nuovo termine nella
        trasformazione dei campi di gauge di teorie non abeliane.
        Tecnicamente, i campi di gauge trasformano nella rappresentazione
        aggiunta di $G$, i cui generatori sono i $(t^b_A)_a^c = i f_a^{bc}$.
        \begin{align*}
            \delta \varphi &= -i (t_R^a)\alpha_a\varphi &\text{campi di materia}\\
            \delta A_{a\mu} &= -i (t_A^b)_a^c \alpha_b A_{c\mu} &\text{campi di gauge.}
        \end{align*}
        Poich\'e i campi di gauge trasformano in modo non banale sotto
        l'azione del gruppo, essi trasportano una carica.
        La lagrangiana cos\`i ottenuta $\lagr(\varphi, \Dimu \varphi)$
        \`e ora invariante per trasformazioni locali.
    \item si completa la lagrangiana con un termine cinetico per i campi di
        gauge, analogamente al termine $F_{\mu\nu}F^{\mu\nu}$ in QED.

        Nel caso non abeliano:
        \begin{align*}
            ([\Dimu, \Dinu]\varphi)' &= \Omega [\Dimu, \Dinu]\varphi = \Omega [\Dimu, \Dinu]
            \Omega^{-1}\varphi'\\
            [\Dimu, \Dinu]\varphi &= (\dimu + A_\mu)(\dinu + A_\nu) \varphi -
            (\mu \leftrightarrow \nu)\\
            &= \underbrace{\dimu \dinu \varphi + A_\nu(\dimu \varphi) + A_\mu(\dinu
            \varphi)}_{\text{simmetrico, si cancella}} + (\dimu A_\nu)\varphi + A_\mu
            A_\nu \varphi - (\mu \leftrightarrow \nu)\\
            &= \underbrace{\{ (\dimu A_\nu - \dinu A_\mu) + [A_\mu, A_\nu]
            \}}_{\mathop{:}= F_{\mu\nu}} \varphi\\
            F_{\mu\nu}' = \Omega F_{\mu\nu} \Omega^{-1}
        \end{align*}
        O, in termini dei campi $A_{a\mu}$
        \begin{equation}
            F_{a\mu\nu} = \dimu A_{a\nu} - \dinu A_{a\mu} - f^{bc}_a
            A_{b\mu}A_{c\nu}.
            \label{eq:fmunu}
        \end{equation}
        Possiamo ora inserire un termine cinetico invariante di gauge e
        definito positivo. Questo perch\'e vogliamo che l'hamiltoniana abbia
        un minimo. Tale termine sar\`a proporzionale, con una costante
        $k$ alla traccia
        \begin{equation*}
            k \tr(F_{\mu\nu}F^{\mu\nu}) = -k F_{a\mu\nu}F^{\mu\nu}_b \tr(t^a
            t^b)
        \end{equation*}
        
        Per un generico gruppo compatto $K^{ab} = \tr(t^a_R t^b_R)$
        \`e definita positiva. Infatti $K^{ab}u_a u_b = \tr( (t^a_R u_a)^2)
        \geq 0$ perch\'e i generatori sono hermitiani.

        Scegliamo allora la base in cui $K^{ab} = C\delta^{ab}$ \`e diagonale e multiplo
        dell'identit\`a. Infine, per analogia con la QED, fissiamo la
        costante $k = 1/4C$.
        \begin{align*}
            -k F_{a\mu\nu}F^{\mu\nu}_b \tr(t^a t^b) &= -kC
            F_{a\mu\nu}F^{a\mu\nu}\\
            &= -\dfrac{1}{4}\{ (\dimu A_{a\nu} - \dinu A_{a\mu})
            (\dimu A^{a\nu} - \dinu A^{a\mu}) + \underbrace{\cdots}_{\text{parte non
            abeliana}}\}
        \end{align*}
\end{enumerate}

Siamo pronti per scrivere la lagrangiana pi\`u generale per una teoria di
gauge, ora che abbiamo una parte invariante locale sotto il gruppo $G$ e un
termine cinetico per i nuovi campi vettoriali. Possiamo ancora fissare il
peso relativo $g^{2}$ di questi due termini.
\begin{equation*}
    \lagr = \lagr(\varphi, \Dimu \varphi) -
    \dfrac{1}{4g^2}F_{a\mu\nu}F^{a\mu\nu}
\end{equation*}
Questo peso relativo ha il significato di costante di accoppiamento tra i
campi $\varphi$ a spin $0$ e $\nicefrac{1}{2}$ e i campi vettoriali. Infatti
ridefinendo i campi $A_\mu$:
\begin{align*}
    A_{a\mu} &\longrightarrow g A_{a\mu}\\
    \lagr & \longrightarrow \lagr(\varphi, \Dimu'\varphi) -
    \dfrac{1}{4}F'_{a\mu\nu}F^{'\mu\nu}\\
    &\text{dove}\\
    \Dimu'\varphi &= (\dimu + i g A_{a\mu} t^{a})\varphi\\
    F'_{a\mu\nu} &= \dimu A_{a\nu} - \dinu A_{a\mu} - gf^{bc}_a
    A_{b\mu}A_{c\nu}
\end{align*}
Quest'ultimo termine mette anche in evidenza il fatto che, in una teoria non
abeliana, compaiono dei termini di interazione tra bosoni di gauge.
\begin{figure}[h]
    \begin{center}
        \begin{fmffile}{3bos}
\begin{fmfgraph*}(100,100)
\fmfleft{i1,i2}
\fmfright{o1}
\fmfdot{v1}
\fmf{boson}{i1,v1}
\fmf{boson}{i2,v1}
\fmf{boson}{o1,v1}
\fmflabel{\small{$g$}}{v1}
\end{fmfgraph*}
\end{fmffile}

    \end{center}
    \caption{Interazione a tre bosoni}
    \label{fig:3bos}
\end{figure}
\end{document}


