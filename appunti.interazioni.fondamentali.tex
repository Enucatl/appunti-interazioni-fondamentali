%        File: appunti.interazioni.fondamentali.tex
%     Created: lun nov 21 03:00  2011 C
% Last Change: lun nov 21 03:00  2011 C
%
\documentclass[italian,a4paper]{article}
\usepackage{babel}
\usepackage{feynmf}
\usepackage{slashed}
\usepackage{booktabs}
\usepackage{amssymb,amsmath,amsthm}
\usepackage{mathrsfs}
\usepackage{nicefrac}
\usepackage[text={6in,9in},centering]{geometry}
\usepackage[utf8x]{inputenc}
\usepackage[T1]{fontenc}
\usepackage{ae,aecompl}
\usepackage[bf,footnotesize]{caption}
\frenchspacing
\pagestyle{plain}

\newtheorem*{theorem}{Teorema}
\newtheorem{exercise}{Esercizio}
\theoremstyle{definition}
\newtheorem*{definition}{Definizione}

\DeclareMathOperator{\tr}{tr}
\newcommand{\lagr}{\ensuremath{\mathscr{L}}}
\newcommand{\lagrbos}{\ensuremath{\mathscr{L}_{\text{bos}}}}
\newcommand{\lagrfer}{\ensuremath{\mathscr{L}_{\text{fer}}}}
\newcommand{\dimu}{\ensuremath{\partial_{\mu}}}
\newcommand{\Dimu}{\ensuremath{D_{\mu}}}
\newcommand{\dinu}{\ensuremath{\partial_{\nu}}}
\newcommand{\Dinu}{\ensuremath{D_{\nu}}}
\newcommand{\zboson}{\ensuremath{\mathrm{Z}}}
\newcommand{\wboson}{\ensuremath{\mathrm{W}}}
\renewcommand{\leq}{\leqslant}
\renewcommand{\theta}{\vartheta}

\title{Teoria delle interazioni fondamentali}
\author{Matteo Abis}
\date{\today}
\begin{document}
\maketitle
\tableofcontents

\section{Teorie di gauge}
\begin{definition}[Teoria di gauge] una teoria quantistica di campi
    invariante sotto trasformazioni locali di un gruppo di Lie, detto gruppo
    \emph{di gauge}. Le trasformazioni sono locali se i parametri dipendono
    dal punto dello spazio-tempo.
\end{definition}

\subsection{Prototipo: l'elettrodinamica quantistica}
\begin{description}
    \item[campi:] un campo spinoriale $\psi(x)$, un campo vettoriale
        $A^{\mu}(x)$;
    \item[trasformazioni di gauge:] sotto l'azione degli elementi del gruppo
        di gauge $U(1)$ i campi trasformano come
        \begin{align*}
            \psi^\prime(x) &= e^{-ie\alpha(x)}\psi(x)\\
            A^{\mu\prime}(x) &= A^{\mu}(x) + \dimu \alpha(x);
        \end{align*}
    \item[rinormalizzabilit\`a:] compaiono nella lagrangiana soltanto
        termini con dimensione $d \leq 4$.
\end{description}

La lagrangiana pi\`u generale compatibile con questi requisiti \`e dunque:
\begin{equation*}
    \mathscr{L_{\text{QED}}} = -\dfrac{1}{4}F_{\mu\nu}F^{\mu\nu} + i
    \bar{\psi}\gamma^{\mu}(\dimu + i e A_{\mu})\psi - m
    \bar{\psi}\psi
\end{equation*}

\subsection{Teorie di gauge non abeliane}\label{teorie_di_gauge}
Discutiamo nel dettaglio la costruzione di una generica teoria di gauge,
seguendo gli stessi passi che ci hanno portato alla formulazione
dell'elettrodinamica quantistica. \`E necessario innanzitutto identificare i
componenti fondamentali della teoria.

\begin{description}
    \item[gruppo di gauge $G$:] deve essere un
        \begin{itemize}
            \item gruppo di Lie. Sia $n$ la sua dimensione;
            \item compatto, perch\'e le rappresentazioni siano unitarie;
            \item semplice, ovvero senza sottogruppi invarianti non banali.
                Questa richiesta non \`e fondamentale e sar\`a eliminata in
                seguito.
        \end{itemize}
    \item[campi di spin \nicefrac{1}{2} e spin $0$:] genericamente indicati
        con il multipletto $\varphi$.
    \item[propriet\`a di trasformazione dei campi: ] il multipletto dei
        campi deve trasformare come una rappresentazione $R$ del gruppo
        $G$. Detti $t^{a}_{R}$ ($a = 1, \dots, n$) i generatori del
        gruppo in tale rappresentazione, e $\alpha_a$  i parametri della
        trasformazione
        \begin{equation*}
            \varphi^\prime(x) = \Omega\varphi = e^{-i \alpha_a t^{a}_{R}}\varphi(x).
        \end{equation*}
        \`E talvolta utile considerare trasformazioni infinitesime
        \begin{equation*}
            \delta\varphi =   -i \alpha_a t^{a}_{R} \varphi.
        \end{equation*}
        Introduciamo infine le costanti di struttura dell'algebra di Lie
        $f^{ab}_{c}$
        \begin{equation*}
            [t^{a}, t^{b}] = if^{ab}_{c}t^{c}
        \end{equation*}
\end{description}

Una volta specificati gli ingredienti, la teoria segue immediatamente
dall'applicazione di una procedura quasi meccanica:
\begin{enumerate}
    \item determinazione della lagrangiana $\lagr(\varphi,
        \dimu\varphi)$ pi\`u generale invariante per il gruppo $G$ sotto
        trasformazioni globali, ovvero indipendenti dal punto dello
        spazio-tempo;
    \item promozione delle trasformazioni globali in trasformazioni locali.
        A questo punto i termini con le derivate non trasformano pi\`u come
        i campi e la lagrangiana non \`e pi\`u invariante:
        \begin{equation}
            (\dimu\varphi)^\prime = (\dimu \Omega)\varphi +
            \Omega(\dimu\varphi)\neq \Omega(\dimu \varphi).
            \label{eq:trasformazione_derivata}
        \end{equation}
        Si introduce dunque una \emph{derivata covariante}, che trasforma
        come i campi, $\Dimu$
        \begin{equation*}
            \Dimu \varphi = (\dimu + i A_{a\mu}t^{a})\varphi
        \end{equation*}
        dove abbiamo introdotto un campo vettoriale reale \emph{di gauge} $A_\mu = i A_{a\mu}t^{a}$,
        che \`e un elemento dell'algebra di Lie del gruppo $G$. Vogliamo
        infatti che questo termine cancelli il primo addendo
        della~\eqref{eq:trasformazione_derivata}, che \`e un elemento
        dell'algebra di Lie.
        Imponendo quindi la legge di trasformazione gi\`a valida per i campi
        \begin{equation*}
            (\Dimu \varphi)^\prime = \Omega \Dimu \varphi\\
        \end{equation*}
        \begin{equation*}
            (\dimu + A^\prime_{\mu})\Omega \varphi = (\dimu \Omega)\varphi +
            \Omega (\dimu \varphi) + A^\prime_\mu\Omega\varphi = \Omega(\dimu
            \varphi) + \Omega A_\mu \varphi\\
        \end{equation*}
        \begin{equation*}
            (A^\prime_\mu\Omega - \Omega A_\mu + \dimu \Omega)\varphi = 0
        \end{equation*}
        otteniamo la legge di trasformazione per i campi di gauge,
        moltiplicando a destra per $\Omega^{-1}$:
        \begin{equation}
            A_{\mu}^\prime = \Omega A_\mu \Omega^{-1} - (\dimu \Omega)
            \Omega^{-1}.
            \label{eq:trasformazione_campi_gauge}
        \end{equation}

        La~\eqref{eq:trasformazione_campi_gauge} si pu\`o capire meglio in
        termini dei campi $A_{a\mu}$ scrivendola per trasformazioni
        infinitesime:
        \begin{align}
            i A^\prime_{a\mu}t^a &= (1 - i \alpha_b t^{b}) A_{c\mu}t^{c}(1 + i
            \alpha_b t^{b}) - [\dimu(1 - i \alpha_a t^{a})(1 + \cdots)]\nonumber\\
            &= i A_{a\mu}t^a + \alpha_b A_{c\mu}[t^b, t^c] + i \dimu \alpha_a
            t^a\nonumber\\
            &= i(A_{a\mu} + \dimu \alpha_a)t^a + i f^{bc}_a t^a\nonumber\\
            A^\prime_{a\mu} &= A_{a\mu} + \dimu \alpha_a + f^{bc}_a \alpha_b
            A_{c\mu}.\label{eq:trasformazione_infinitesima_gauge}
        \end{align}
        Vediamo dunque che, rispetto al caso abeliano dell'elettrodinamica
        quantistica, si introduce un nuovo termine nella
        trasformazione dei campi di gauge di teorie non abeliane.
        Tecnicamente, i campi di gauge trasformano nella rappresentazione
        aggiunta di $G$, i cui generatori sono i $(t^b_A)_a^c = i f_a^{bc}$.
        \begin{align*}
            \delta \varphi &= -i (t_R^a)\alpha_a\varphi &\text{campi di materia}\\
            \delta A_{a\mu} &= -i (t_A^b)_a^c \alpha_b A_{c\mu} &\text{campi di gauge.}
        \end{align*}
        Poich\'e i campi di gauge trasformano in modo non banale sotto
        l'azione del gruppo, essi trasportano una carica.
        La lagrangiana cos\`i ottenuta $\lagr(\varphi, \Dimu \varphi)$
        \`e ora invariante per trasformazioni locali.
    \item si completa la lagrangiana con un termine cinetico per i campi di
        gauge, analogamente al termine $F_{\mu\nu}F^{\mu\nu}$ in QED.

        Anche nel caso non abeliano abbiamo che $([\Dimu,
        \Dinu]\varphi)^\prime = \Omega [\Dimu, \Dinu]\varphi$. Infatti si
        vede che il commutatore in realt\`a non \`e un operatore
        differenziale, e la trasformazione viene solo dal campo. Quindi il
        fattore moltiplicativo $[\Dimu, \Dinu]$ deve essere invariante:
        \begin{align*}
            [\Dimu, \Dinu]\varphi &= (\dimu + A_\mu)(\dinu + A_\nu) \varphi -
            (\mu \leftrightarrow \nu)\\
            &= \underbrace{\dimu \dinu \varphi + A_\nu(\dimu \varphi) + A_\mu(\dinu
            \varphi)}_{\text{simmetrico, si cancella}} + (\dimu A_\nu)\varphi + A_\mu
            A_\nu \varphi - (\mu \leftrightarrow \nu)\\
            &= \underbrace{\{ (\dimu A_\nu - \dinu A_\mu) + [A_\mu, A_\nu]
            \}}_{\mathop{:}= F_{\mu\nu}} \varphi\\
            F_{\mu\nu}^\prime = \Omega F_{\mu\nu} \Omega^{-1}
        \end{align*}
        O, in termini dei campi $A_{a\mu}$
        \begin{equation}
            F_{a\mu\nu} = \dimu A_{a\nu} - \dinu A_{a\mu} - f^{bc}_a
            A_{b\mu}A_{c\nu}.
            \label{eq:fmunu}
        \end{equation}
        Possiamo ora inserire un termine cinetico invariante di gauge e
        definito positivo. Questo perch\'e vogliamo che l'hamiltoniana abbia
        un minimo. Tale termine sar\`a proporzionale, con una costante
        $k$ alla traccia
        \begin{equation*}
            k \tr(F_{\mu\nu}F^{\mu\nu}) = -k F_{a\mu\nu}F^{\mu\nu}_b \tr(t^a
            t^b)
        \end{equation*}
        
        Per un generico gruppo compatto $K^{ab} = \tr(t^a_R t^b_R)$
        \`e definita positiva. Infatti $K^{ab}u_a u_b = \tr( (t^a_R u_a)^2)
        \geq 0$ perch\'e i generatori sono hermitiani.

        Scegliamo allora la base in cui $K^{ab} = C\delta^{ab}$ \`e diagonale e multiplo
        dell'identit\`a. Infine, per analogia con la QED, fissiamo la
        costante $k = 1/4C$.
        \begin{align*}
            -k F_{a\mu\nu}F^{\mu\nu}_b \tr(t^a t^b) &= -kC
            F_{a\mu\nu}F^{a\mu\nu}\\
            &= -\dfrac{1}{4}\{ (\dimu A_{a\nu} - \dinu A_{a\mu})
            (\dimu A^{a\nu} - \dinu A^{a\mu}) + \underbrace{\cdots}_{\text{parte non
            abeliana}}\}
        \end{align*}
\end{enumerate}

Siamo pronti per scrivere la lagrangiana pi\`u generale per una teoria di
gauge, ora che abbiamo una parte invariante locale sotto il gruppo $G$ e un
termine cinetico per i nuovi campi vettoriali. Possiamo ancora fissare il
peso relativo $g^{2}$ di questi due termini.
\begin{equation*}
    \lagr = \lagr(\varphi, \Dimu \varphi) -
    \dfrac{1}{4g^2}F_{a\mu\nu}F^{a\mu\nu}
\end{equation*}
Questo peso relativo ha il significato di costante di accoppiamento tra i
campi $\varphi$ a spin $0$ e $\nicefrac{1}{2}$ e i campi vettoriali. Infatti
ridefinendo i campi $A_\mu$:
\begin{align*}
    A_{a\mu} &\longrightarrow g A_{a\mu}\\
    \lagr & \longrightarrow \lagr(\varphi, \Dimu^\prime\varphi) -
    \dfrac{1}{4}F^\prime_{a\mu\nu}F^{\prime\mu\nu}\\
    &\text{dove}\\
    \Dimu^\prime\varphi &= (\dimu + i g A_{a\mu} t^{a})\varphi\\
    F^\prime_{a\mu\nu} &= \dimu A_{a\nu} - \dinu A_{a\mu} - gf^{bc}_a
    A_{b\mu}A_{c\nu}
\end{align*}
Quest'ultimo termine mette anche in evidenza il fatto che, in una teoria non
abeliana, compaiono dei termini di interazione tra bosoni di gauge.
\begin{figure}[h]
    \begin{center}
        \begin{fmffile}{3bos}
\begin{fmfgraph*}(100,100)
\fmfleft{i1,i2}
\fmfright{o1}
\fmfdot{v1}
\fmf{boson}{i1,v1}
\fmf{boson}{i2,v1}
\fmf{boson}{o1,v1}
\fmflabel{\small{$g$}}{v1}
\end{fmfgraph*}
\end{fmffile}

        \begin{fmffile}{4bos}
\begin{fmfgraph*}(100,100)
\fmfleft{i1,i2}
\fmfright{o1,o2}
\fmfdot{v1}
\fmf{boson}{i1,v1}
\fmf{boson}{i2,v1}
\fmf{boson}{o1,v1}
\fmf{boson}{o2,v1}
\fmflabel{\small{$g^2$}}{v1}
\end{fmfgraph*}
\end{fmffile}

    \end{center}
    \caption{interazioni a tre e quattro bosoni, che derivano dai nuovi
    termini nella lagrangiana per teorie non abeliane.}
    \label{fig:3bos}
\end{figure}
\subsection{Esempio: la cromodinamica quantistica}\label{qcd}
\begin{description}
    \item[gruppo:] $G = SU(3)$
    \item[campi e trasformazioni:] spinori di Dirac $q$ che trasformano con la
        rappresentazione fondamentale $3$ di $SU(3)$. Quindi $q$ \`e un
        oggetto di dimensione $3$ e possiamo scrivere esplicitamente
        l'indice \emph{di colore} $c = 1, 2, 3$. 
        \begin{equation*}
            q_c^\prime = e^{-i \alpha_a \lambda^a}q_c.
        \end{equation*} I generatori $\lambda_a$ sono le otto matrici di
        Gell-Mann.
\end{description}
Con la procedura ora descritta si ricava subito la lagrangiana della QCD:
\begin{enumerate}
    \item scriviamo la lagrangiana pi\`u generale con invarianza globale
        \begin{equation*}
            \lagr = i \bar{q}_c\gamma^\mu\dimu q_c - m
            \bar{q}_c q_c;
        \end{equation*}
    \item rendiamo l'invarianza locale introducendo la derivata covariante e
        i campi di gauge $G_{a\mu}$:
        \begin{equation*}
            \Dimu q = (\dimu + i g_s G_{a\mu}\lambda^a)q
        \end{equation*}
\item completando con i termini cinetici:
    \begin{equation*}
        \lagr = - \dfrac{1}{4}G_{a\mu\nu}G^{a\mu\nu} + i \bar q
        \gamma^\mu(\dimu + i g_s G_{a\mu}\lambda^a)q - m \bar q q
    \end{equation*}
\item resta solo da estendere al caso di sei sapori di quark $f = u$,
    $d$,\dots,$t$.
    \begin{equation*}
        \lagr_{\text{QCD}} = - \dfrac{1}{4}G_{a\mu\nu}G^{a\mu\nu} + \sum_f[i
        \bar q_f
        \gamma^\mu(\dimu + i g_s G_{a\mu}\lambda^a)q_f - m \bar q_f q_f]         
    \end{equation*}
\end{enumerate}
\section{Rottura spontanea di simmetria}
\subsection{Il teorema di Goldstone}\label{goldstone}
Con la rottura spontanea di una simmetria globale in una teoria quantistica
di campi, compaiono particelle di spin $0$ e massa nulla, detti bosoni di
Goldstone.
\subsection*{Esempio: campo scalare complesso}
La lagrangiana per il campo scalare complesso, con una simmetria globale
$U(1)$ \`e:
\begin{align*}
    \lagr &= (\partial_\mu \varphi)^\dagger (\partial^\mu \varphi) -
    V(|\varphi|^2)\\
    &= (\partial_\mu \varphi)^\dagger (\partial^\mu \varphi) -
    m^2\varphi^\dagger\varphi - \lambda (\varphi^\dagger \varphi)^2
\end{align*}
Dove $\lambda > 0$ perch\'e l'energia abbia un minimo. Ora, con $m^2 > 0$
otteniamo la solita teoria del campo scalare complesso, mentre il caso
$m^2 < 0$ \`e pi\`u interessante.

Consideriamo infatti il caso $m^2 < 0$.
Scriviamo il campo con le sue componenti reali. In termini di queste
componenti, la simmetria $U(1)$ diventa una simmetria per rotazioni $SO(2)$.
\begin{align*}
    \varphi &= \dfrac{1}{\sqrt{2}}(\varphi_1 + i \varphi_2)\\
    \begin{pmatrix}
        \varphi_1^\prime\\
        \varphi_2^\prime
    \end{pmatrix} &= 
    \begin{pmatrix}
        \cos \alpha & \sin \alpha \\
        -\sin \alpha & \cos \alpha
    \end{pmatrix}
    \begin{pmatrix}
        \varphi_1\\
        \varphi_2
    \end{pmatrix}
\end{align*}
Cerchiamo i minimi dell'hamiltoniana $\mathscr H = |\dot\varphi|^2 + |\nabla
\varphi|^2 + V(|\varphi|^2)$. La parte con le derivate si annulla per
$\varphi$ costante, cerchiamo quindi i minimi del potenziale.
\begin{align*}
    V &=\dfrac{1}{2}(\varphi_1^2 + \varphi_2^2) +
    \dfrac{\lambda}{4}(\varphi_1^2 + \varphi_2^2)^2\\
    \dfrac{\partial V}{\partial \varphi_i} &= m^2 \varphi_i ^ 2 +
    \lambda(\varphi_1^2 + \varphi_2^2)\varphi_i\\
    &= \varphi_i (m^2 + \lambda(\varphi_1^2 + \varphi_2^2)).
\end{align*}
Quindi le derivate si annullano per 
\begin{enumerate}
    \item $\varphi_1 = \varphi_2 = 0$
    \item $\varphi_1^2 + \varphi_2^2 = - \dfrac{m^2}{\lambda} = \mathop: v$
\end{enumerate}
Calcolando la matrice delle derivate
seconde si verifica facilmente che solo la seconda possibilit\`a corrisponde
a un minimo, e che gli autovalori in questo caso sono $0$ e $1$. Inoltre,
parametrizziamo i campi attorno al minimo
\begin{align*}
    v_1 &=v \cos \theta\\
    v_2 &=v \sin \theta.
\end{align*}
Scegliendo un minimo abbiamo una rottura spontanea di simmetria. Espandiamo
la lagrangiana intorno a questo minimo:
\begin{equation*}
    V = V(2.)
    + \underbrace{\dfrac{\partial V}{\partial \varphi_1}\rvert_{2.}(\varphi_1 - v_1)
    + \dfrac{\partial V}{\partial \varphi_2}\rvert_{2.}(\varphi_2 -
    v_2)}_{= 0\text{ nel minimo}}
    + \dfrac{1}{2}\dfrac{\partial^2 V}{\partial \varphi_i\partial \varphi_j}|_{2.}
    \underbrace{(\varphi_i - v_i)}_{=\mathop:\varphi_i^\prime}\underbrace{(\varphi_j -
    v_j)}_{=\mathop:\varphi_j^\prime}
\end{equation*}
Con le ridefinizioni dei campi $\varphi_i^\prime$ possiamo scrivere la lagrangiana
come
\begin{align*}
    \lagr &= 
    \dfrac{1}{2}\partial_\mu \varphi_1^\prime\partial^\mu \varphi_1^\prime +
    \dfrac{1}{2}\partial_\mu \varphi_2^\prime\partial^\mu \varphi_2^\prime +
    \dfrac{1}{2}2\lambda v^2
    \begin{pmatrix}
        \varphi_1^\prime & \varphi_2^\prime
    \end{pmatrix}
    \begin{pmatrix}
        \cos^2 \theta & \cos\theta \sin\theta\\
        \cos\theta \sin\theta & \sin^2 \theta
    \end{pmatrix}
    \begin{pmatrix}
        \varphi_1^\prime\\
        \varphi_2^\prime
    \end{pmatrix}\\
    &= 
    \dfrac{1}{2}\partial_\mu \varphi_1^\prime\partial^\mu \varphi_1^\prime +
    \dfrac{1}{2}\partial_\mu \varphi_2^\prime\partial^\mu \varphi_2^\prime +
    \lambda v^2(\cos\theta \varphi_1^\prime + \sin\theta \varphi_2^\prime)^2
\end{align*}
Con un cambio di variabili finale possiamo ridefinire i campi e la
lagrangiana con
\begin{align*}
    \begin{pmatrix}
        \varphi_1\\
        \varphi_2
    \end{pmatrix} &= 
    \begin{pmatrix}
        \cos \theta & \sin \theta \\
        -\sin \theta & \cos \theta
    \end{pmatrix}
     \begin{pmatrix}
        \varphi_1^\prime\\
        \varphi_2^\prime
    \end{pmatrix}\\
\lagr &=
    \dfrac{1}{2}(\partial_\mu \varphi_1\partial^\mu \varphi_1 +
    \partial_\mu \varphi_2\partial^\mu \varphi_2) -
    \lambda v^2 \varphi_1^2
\end{align*}
In questi termini, la lagrangiana descrive due campi scalari: $\varphi_1$ con massa
$m_1 = 2\lambda v^2$ e $\varphi_2$ con massa nulla, detto bosone di
Goldstone. Possiamo ora enunciare il teorema in generale.
\begin{theorem}[di Goldstone]
    \label{teorema_goldstone}
    \hspace*{\fill}
    \begin{itemize}
        \item Sia $\lagr(\varphi, \chi)$ una lagrangiana con campi reali di spin 0
    $\varphi$ e campi di spin \nicefrac{1}{2} $\chi$;
        \item $\lagr$ invariante globale sotto l'azione di un gruppo $G$.
        \item $\varphi_i = v_i$ configurazione costante che minimizza
            l'energia;
        \item infine diciamo $H < G$ la simmetria residua, ovvero il
            sottogruppo di $G$ che lascia invariata la configurazione di
            equilibrio.
    \end{itemize}
    Allora la matrice delle derivate seconde di $V$ ha esattamente
    $\dim(G) - \dim(H)$ autovalori nulli, che corrispondono a particelle di
    massa nulla e spin $0$ (bosoni di Goldstone).
    \begin{proof}
        La lagrangiana \`e:
        \begin{equation*}
            \lagr = \dfrac{1}{2}\partial_\mu \varphi_i\partial^\mu
            \varphi_i- V(\varphi) + \underbrace{\cdots}_{\text{dipendenza da
            } \chi}
        \end{equation*}
        e, nella configurazione di minimo vale:
        \begin{equation*}
            0 = \delta V = \dfrac{\partial V}{\partial \varphi_i}\delta \varphi_i = 
            \dfrac{\partial V}{\partial \varphi_i}(-i \alpha_a
            t^a_{ij}\varphi_j) \quad \forall \alpha_a
        \end{equation*}
        e deve essere anche nulla l'espressione:
        \begin{equation*}
             \dfrac{\partial V}{\partial \varphi_i}t^a_{ij}\varphi_j = 0.
        \end{equation*}
        Derivando rispetto a $\varphi_k$
        \begin{equation*}
            \dfrac{\partial^2 V}{\partial\varphi_i
            \partial\varphi_k}t^a_{ij}\varphi_j +
            \underbrace{\dfrac{\partial V}{\partial \varphi_i}}_{=0\text{
            nel minimo}}t^a_{ik} = 0
        \end{equation*}
        Quindi rimane:
        \begin{equation*}
            \dfrac{\partial^2 V}{\partial\varphi_i
            \partial\varphi_k}t^a_{ij}v_j = 0 
        \end{equation*}
        da cui si legge che $t^a_{ij}v_j$ \`e un autovettore relativo
        all'autovalore 0. Contiamo correttamente gli autovettori indipendenti. Ordiniamo i generatori mettendo prima i
        generatori del gruppo $H$: $t^a_{ij} = \{t^1,\ldots,t^{\dim(H)},
        \ldots t^{\dim(G)} \}$.

        Se $a = 1,\ldots, \dim(H)$ allora $h = e^{-i \alpha_a t^a} \in H$ e,
        per ipotesi, $hv = v$. Per una trasformazione infinitesima dunque
        $(1 - i\alpha_a t^a) v = v$, da cui si legge $t^a v = 0.$.
        Restano dunque $\dim(G) - \dim(H)$ autovettori non nulli linearmente
        indipendenti.

        La matrice delle derivate seconde del potenziale diventa ovviamente
        la matrice di massa. Espandendo intorno alla configurazione di
        equilibrio $\varphi_i = v_i$ e ridefinendo $\varphi_i^\prime = \varphi_i -
        v_i$, otteniamo le equazioni del moto:
        \begin{equation*}
            \left( \box \delta_{ij} + \dfrac{\partial^2 V}{\partial
            \varphi_i \partial \varphi_j}|_{\varphi = v} \right) \varphi^\prime_j
            = 0
        \end{equation*}
        Da cui si vede che il sistema ha $\dim(G) - \dim(H)$ particelle a
        spin e massa nulli, dette appunto bosoni di Goldstone.
    \end{proof}
\end{theorem}
\subsection*{Esempio: i mesoni pseudoscalari leggeri}
Non sono mai stati osservati i bosoni di Goldstone a spin $0$ e massa nulla.
Tuttavia la teoria si pu\`o applicare a un'approssimazione della QCD.
Considerando solo i quark leggeri $q = (u, d, s)$, la lagrangiana ricavata nel paragrafo~\ref{qcd}:
    \begin{equation*}
        \lagr = - \dfrac{1}{4}G_{a\mu\nu}G^{a\mu\nu} + i \bar q
        \gamma^\mu\Dimu q - \bar q m q
    \end{equation*}
Se trascuriamo le masse dei quark leggeri, otteniamo una lagrangiana
invariante sotto un gruppo di simmetria pi\`u ampio. Infatti scompare
l'unico termine che mescola le chiralit\`a \emph{left}e \emph{right}, che
ora possono trasformare con due fasi indipendenti. Il gruppo diventa quindi
$G = SU(3)_L \times SU(3)_R$.

Tuttavia, non esistono multipletti dinamici con masse molto simili. Questo
fa pensare che la simmetria sia spontaneamente rotta. Il ruolo dei campi
scalari in questa teoria \`e giocato dai bilineari fermionici $\bar q q = \bar u u +
\bar d d + \bar s s$.

Una trattazione non perturbativa mostra in effetti che la configurazione di
energia minima non si trova nell'origine $\langle 0 \vert \bar q q \vert 0
\rangle \neq 0$. Qual \`e la simmetria residua $H$? Un elemento di questo
sottogruppo dovrebbe lasciare invariato:
\begin{equation*}
    \langle 0 \vert \bar q_R q_L + \text{h.c.} \vert 0 \rangle
    \xrightarrow{G} 
    \langle 0 \vert \bar q_R \Omega^\dagger_R \Omega_L q_L + \text{h.c.}
    \vert 0 \rangle.
\end{equation*}
Ci\`o accade se e solo se le fasi con cui trasformano le parti
\emph{left} e \emph{right} sono uguali. $\alpha_{aL} = \alpha_{aR} =\mathop:
\alpha_a$. Risulta dunque che il gruppo $H$ \`e $SU(3)$. Questo gruppo ha
otto generatori, quindi la teoria avr\`a $8 + 8 - 8 = 8$ bosoni di Goldstone.
Queste otto particelle con spin $0$ e massa piccola sono gli otto mesoni
pseudoscalari leggeri $\pi$, $\mathrm{K}$ e $\eta$.
\subsection{Il meccanismo di Higgs}
Il meccanismo di Higgs consente di dare massa ai bosoni di gauge attraverso
la rottura spontanea di una simmetria continua \emph{locale}. In questo
modo, la teoria conserva l'invarianza di gauge.
\subsection*{Esempio: modello di Higgs abeliano}
\`E la versione locale dell'esempio visto nel
paragrafo~\ref{goldstone} per il teorema di Goldstone. Sia $\varphi$ un
campo scalare complesso e $G = U(1)$ il gruppo di invarianza locale. Le
trasformazioni sono dunque
\begin{align*}
    \varphi^\prime(x) &= e^{-i \alpha(x)}\varphi(x)\\
    A_\mu^\prime(x) &= A_\mu(x) - \dfrac{1}{g}\partial_\mu \alpha(x).
\end{align*}
Ricordiamo anche la lagrangiana
\begin{align*}
    \lagr &= -\dfrac{1}{4}F_{\mu\nu}F^{\mu\nu} + (D_\mu\varphi)^\dagger(D^\mu
    \varphi) - V(|\varphi|^2)\\
    V &= m^2 \varphi^\dagger \varphi + \lambda (\varphi^\dagger \varphi)^2.
\end{align*}
Con gli stessi passaggi del caso globale, con $\lambda > 0$ e $m^2 <0$,
troviamo la configurazione di minima energia, ovvero il valore di
aspettazione del vuoto
\begin{equation*}
    \langle 0 \vert \varphi \vert 0 \rangle \mathop:= \langle \varphi
    \rangle = \dfrac{v}{\sqrt{2}}e^{i\theta}
\end{equation*}
essendo $\langle A_\mu \rangle = 0$ per mantenere l'invarianza di Lorentz.
Mandiamo tutti i minimi nell'origine delle nuove coordinate, usando una
rappresentazione polare con un modulo e una fase (due campi reali):
\begin{equation*}
    \varphi(x) = \dfrac{v +
    \sigma(x)}{\sqrt{2}}e^{i\theta}e^{i\frac{\xi(x)}{v}}
\end{equation*}
si vede che ho il minimo per $\sigma = \xi = 0$, e quindi $\varphi =
ve^{i\theta}/\sqrt{2}$. Scriviamo la lagrangiana in termini di questi due
nuovi campi:
\begin{align*}
    \varphi^\dagger \varphi &= \dfrac{(v + \sigma)^2}{2}\\
    V &= \dfrac{m^2}{2}(v + \sigma)^2 + \dfrac{\lambda}{4}(v + \sigma)^4\\
    \\
    \Dimu \varphi &= (\dimu + i g A_{\mu}) \dfrac{v +
    \sigma}{\sqrt{2}}e^{i\theta}e^{i\frac{\xi}{v}}\\
    &= e^{i\theta}e^{i\frac{\xi}{v}}
    \left\{
    \dfrac{1}{\sqrt{2}}\dimu \sigma 
    + i g A_\mu \dfrac{v + \sigma}{\sqrt{2}}
    + \dfrac{v + \sigma}{\sqrt{2}} i\dfrac{\dimu \xi}{v}\right\}\\
    &= e^{i\theta}e^{i\frac{\xi}{v}}
    \left\{
    \dfrac{\dimu \sigma}{\sqrt{2}}
    + i g \dfrac{v + \sigma}{\sqrt{2}} \left[ A_\mu + \dfrac{\dimu \xi}{gv}\right]
    \right\}\\
    \\
    \lagr &= -\dfrac{1}{4}F_{\mu\nu}F^{\mu\nu}
    + \dfrac{1}{2}\partial_\mu \sigma \partial^\mu \sigma
    + g^2 \dfrac{(v + \sigma)^2}{2}\left[ A_{\mu} + \dfrac{\dimu
    \xi}{gv}\right]^2 - V(\sigma)
\end{align*}
Ridefiniamo $A_\mu = A_{\mu} + \dimu \xi / gv$. Non \`e una trasformazione
di gauge perch\'e non coinvolge il campo $\varphi$.
\begin{equation*}
    \lagr =  -\dfrac{1}{4}F_{\mu\nu}F^{\mu\nu}
    + \dfrac{1}{2}\partial_\mu \sigma \partial^\mu \sigma
    + \dfrac{g^2}{2}(v + \sigma)^2 A_{\mu}A^{\mu} - V(\sigma)
\end{equation*}
Leggiamo la fisica da questa lagrangiana:
\begin{itemize}
    \item il campo $\xi$ \`e sparito, non fa parte della fisica;
    \item \`e comparso un termine di massa per il campo $A_\mu$, con
        $m_A^2 = g^2 v^2$;
    \item $\xi$ \`e sparito ma la teoria ha lo stesso numero di gradi di
        libert\`a di prima. Nel caso $m^2 > 0$ infatti, $\varphi$ e
        $A_\mu$ trasportano ciascuno due gradi di libert\`a. Nel caso
        $m^2 < 0$ invece $A_\mu$ ha una massa e quindi tre gradi di
        libert\`a, con un solo grado di libert\`a per il campo reale
        $\sigma$.
\end{itemize}

Notiamo che con la quantizzazione della teoria ci sono delle differenze
rispetto al caso classico. Infatti lo stato di vuoto della teoria
quantistica $\vert 0 \rvert$, che corrisponde al minimo dell'energia, \`e
unico. Al contrario, nella teoria classica abbiamo infiniti minimi
raggiungibili con trasformazioni di $G$. Quindi dovremmo avere $U\vert 0
\rvert \neq \vert 0 \rvert$ e quindi la carica del vuoto sarebbe $Q \vert 0
\rvert \neq 0$.

Questa apparente contraddizione si risolve osservando che nella teoria
quantistica il segnale di rottura spontanea di simmetria \`e molto
pi\`u sottile: non si pu\`o infatti costruire l'operatore $Q$ perch\'e
$\int d^3 x j^0 \rightarrow \infty$.
\section{La lagrangiana della teoria elettrodebole}
\subsection{Parte bosonica}
\begin{description}
    \item[gruppo di gauge:]
$G = SU(2) \times U(1)$. I generatori $T^1,T^2, T^3, Y$ soddisfano alle regole di commutazione
\begin{align*}
    [T^a, T^b] &= i \varepsilon^{abc}T^c\\
    [T^a, Y] &= 0.
\end{align*}
\item[campi di materia a spin $0$:] devono realizzare la rottura spontanea
    di $SU(2)\times U(1)$ in modo da dare massa ai bosoni $\wboson^{\pm}$ e
    $\zboson$, con un'invarianza residua $U(1)_{em}$ in modo da lasciare
    il fotone a massa nulla.

    La scelta pi\`u semplice \`e un doppietto complesso di Higgs con
    ipercarica $Y = \nicefrac{1}{2}$, che trasforma sotto $G$ come la
    rappresentazione $(2, \nicefrac{1}{2})$:
    \begin{align*}
        \varphi &= 
        \begin{pmatrix}
            \varphi^+ \\
            \varphi^0
        \end{pmatrix}\\
        \varphi^\prime &= e^{-i \alpha_a \frac{\sigma^a}{2}}e^{-i\beta
        \frac{1}{2}}\varphi
    \end{align*}
    Non \`e ora restrittivo, percorrendo le valli di minimo, raggiungere un
    minimo in cui $\langle \varphi \rangle = (0, v/\sqrt{2})$. Verifichiamo
    che c'\`e un'invarianza residua. Per una trasformazione infinitesima
    \begin{equation*}
        (-i \alpha_a \dfrac{\sigma^a}{2} -i \beta \dfrac{1}{2})
        \begin{pmatrix}
            0 \\
            v/\sqrt{2}
        \end{pmatrix} = 
        -\dfrac{i}{2}
        \begin{pmatrix}
            \alpha_3 + \beta & \alpha_1 -i \alpha_2\\
            \alpha_1 + i \alpha_2 & -\alpha_3 + \beta
        \end{pmatrix}
        \begin{pmatrix}
            0\\
            v/\sqrt{2}
        \end{pmatrix}
        =
        -\dfrac{iv}{2\sqrt{2}}
        \begin{pmatrix}
            \alpha_1 - i \alpha_2\\
            -\alpha_3 + \beta
        \end{pmatrix}
    \end{equation*}
    e tale variazione \`e nulla evidentemente se e solo se $\alpha_1 =
    \alpha_2 = 0$ e $\alpha_3 = \beta =\mathop: \alpha$. Quindi c'\`e un
    sottogruppo $U(1)$ che lascia invariata la configurazione di minimo. Il
    generatore corrispondente \`e
    \begin{equation*}
        Q = T^3 + Y = \dfrac{\sigma^3}{2} + \dfrac{1}{2} = 
        \begin{pmatrix}
            1 & 0\\
            0 & 0
        \end{pmatrix}.
    \end{equation*}
\end{description}
Dati questi ingredienti, con la procedura descritta nel
paragrafo~\ref{teorie_di_gauge}, \`e automatico scrivere la lagrangiana. Ad
ognuno dei generatori del gruppo corrisponde un campo di gauge, con la
corrispondente legge di trasformazione:
\begin{align*}
    T_a &\rightarrow W_{a\mu}\\
    Y &\rightarrow B_\mu\\
    \\
    \delta W_{a\mu} &= \varepsilon_{abc}\alpha^b W^c_\mu +
    \dfrac{\dimu \alpha_a}{g}\\
    \delta B_\mu &= \dfrac{\dimu \beta}{g^\prime}.
\end{align*}
Scriviamo subito la lagrangiana
\begin{align*}
    \lagrbos &= -\dfrac{1}{4}W_{a\mu\nu}W^{a\mu\nu} -
    \dfrac{1}{4}B_{\mu\nu}B^{\mu\nu} +
    (D_\mu \varphi)^\dagger (D^\mu \varphi) - V(\varphi)\\
    \\
    \Dimu &= \dimu + i g W_{a\mu}\dfrac{\sigma^a}{2} + i
    \dfrac{g^\prime}{2}B_\mu\\
    V(\varphi) &= \mu^2 \varphi^\dagger \varphi + \lambda (\varphi^\dagger
    \varphi)^2
\end{align*}
Scegliamo, come nell'esempio, il minimo con $\langle \varphi \rangle = (0,
v/\sqrt{2})$.
Bisogna scegliere una parametrizzazione per $\varphi$ per leggere masse e
accoppiamenti
\begin{equation*}
    \varphi(x) = e^{i\xi_a(x)\frac{\sigma^a}{v}}
    \begin{pmatrix}
        0\\
        \dfrac{v + h(x)}{\sqrt{2}}
    \end{pmatrix}.
\end{equation*}
Come nel caso abeliano, gli $\xi_a$ scompaiono, quindi scriviamo
direttamente
\begin{align*}
    \varphi(x) &= 
    \begin{pmatrix}
        0 \\
        \dfrac{v + h(x)}{\sqrt{2}}
    \end{pmatrix}\\
    V(\varphi) &= \dfrac{\mu^2}{2}(v + h)^2 + \dfrac{\lambda}{4}(v + h)^4\\
    &= \text{cost.} + \underbrace{\lambda v^2 h^2}_{\text{termine di massa}} + \mathcal{O}(h^3)
\end{align*}
Da cui si legge subito che il bosone di Higgs ha una massa $m_h = 2\lambda
v^2$. Il termine cinetico d\`a ora una massa ai bosoni vettori
\begin{align*}
    \Dimu &= \dimu + i g W_{a\mu}\dfrac{\sigma^a}{2} + i
    \dfrac{g^\prime}{2}B_\mu\\
    \Dimu \varphi &= \left\{ \dfrac{\dimu h}{\sqrt{2}}
    + i g W_{a\mu}\dfrac{\sigma^a}{2}\dfrac{v + h}{\sqrt{2}}
    + i g^\prime \dfrac{v + h}{2 \sqrt{2}} B_\mu \right\}
    \begin{pmatrix}
        0\\
        1
    \end{pmatrix}\\
    &= \left\{ \dfrac{\dimu h}{\sqrt{2}}
    + \dfrac{i}{2}\dfrac{v + h}{\sqrt{2}}(g W_{a\mu}\sigma^a + g^\prime B_\mu) \right\}
    \begin{pmatrix}
        0\\
        1
    \end{pmatrix}\\
    \\
    D_\mu \varphi^\dagger D^\mu \varphi &=
    \dfrac{1}{2}\partial_\mu h
    \partial^\mu h +
    \underbrace{
    \begin{pmatrix}
        0 & 1
    \end{pmatrix}
    \dfrac{(v + h)^2}{8}
    (g W_a \sigma^a + g^\prime B)^2
    \begin{pmatrix}
        0\\
        1
    \end{pmatrix}
    }_{\text{seleziona l'elemento } 2,2 \text{ della matrice}}\\
    &= 
    \dfrac{1}{2}\partial_\mu h
    \partial^\mu h +
    \dfrac{(v + h)^2}{8}
    [g W_a \sigma^a + g^\prime B]^2_{22}.
\end{align*}
Calcoliamo l'elemento di matrice richiesto
\begin{align*}
    [g W_a \sigma^a + g^\prime B]^2_{22} &= 
    \begin{pmatrix}
        gW_3 + g^\prime B & g(W_1 - iW_2)\\
        g(W_1 + i W_2) & - g W_3 + g^\prime B
    \end{pmatrix}^2_{22}\\
    &= 
    \begin{pmatrix}
        g (W_1 + i W_2) & - g W_3 + g^\prime B
    \end{pmatrix}
    \begin{pmatrix}
        g(W_1 - i W_2)\\
        - g W_3 + g^\prime B
    \end{pmatrix}\\
    &= g^2(W_1^2 + W_2^2) + (gW_3 - g^\prime B)^2.
\end{align*}
\begin{definition}
    [angolo di Weinberg]
    Si definisce l'angolo $\theta$ di Weinberg con
    \begin{equation*}
        \tan \theta \mathop:= \dfrac{g^\prime}{g}
    \end{equation*}
\end{definition}
Introduciamo delle nuove combinazioni dei campi
\begin{align}
\label{eq:rotazione_weinberg}
    W^{\pm}_\mu &= \dfrac{W_1 \mp i W_2}{\sqrt{2}}\nonumber\\
    Z_\mu &= \cos \theta W_{3\mu} - \sin \theta B_{\mu}\\
    A_\mu &= \sin \theta W_{3\mu} + \cos \theta B_\mu \nonumber
\end{align}
In termini di questi quattro campi possiamo riscrivere:
\begin{equation*}
    D_\mu \varphi^\dagger D^\mu \varphi =
    \dfrac{1}{2}\partial_\mu h \partial^\mu h +
    \dfrac{g^2(v + h)^2}{4}
    W_\mu^+W^{-\mu} +
    (g^2 + g^{\prime2})
    \dfrac{(v + h)^2}{8}
    Z_\mu Z^\mu.
\end{equation*}
Da questa lagrangiana possiamo leggere le masse:
\begin{align*}
    m_\wboson^2 &= \dfrac{g^2 v^2}{4}\\
    m_\zboson^2 &= \dfrac{(g^2 + g^{\prime 2})v^2}{4}\\
    m_A^2 &= 0\\
    m_h^2 &= 2\lambda v^2
\end{align*}
Si noti che, come conseguenza diretta del fatto che il campo di Higgs
trasforma come un doppietto, abbiamo una relazione tra le masse dei bosoni
$m_\wboson = m_\zboson \cos \theta$. Tale relazione \`e verificata con una precisione
sperimentale del per mille.
\begin{exercise}
    Trovare le interazioni che derivano dalla lagrangiana e dimostrare che
    vale
    \begin{equation*}
        g \sin \theta = g^\prime \cos \theta = e.
    \end{equation*}
\end{exercise}
\begin{exercise}
    Calcolare la larghezza del decadimento $h \rightarrow \wboson^+ \wboson^-$.
\begin{figure}[h]
    \begin{center}
        \begin{fmffile}{hWW}
\begin{fmfgraph*}(100,100)
\fmfleft{i1}
\fmfright{o2,o1}
\fmfdot{v1}
\fmf{dots}{i1,v1}
\fmf{boson,label=$p_2$}{v1,o2}
\fmf{boson,label=$p_1$}{v1,o1}
\fmflabel{\small{$h$}}{i1}
\fmflabel{\small{$W^-$}}{o1}
\fmflabel{\small{$W^+$}}{o2}
\end{fmfgraph*}
\end{fmffile}

    \end{center}
    \caption{grafico di Feynman al primo ordine per il decadimento $h \rightarrow \wboson^+ \wboson^-$.}
    \label{fig:hWW}
\end{figure}
\begin{align*}
    d\Gamma &= \dfrac{1}{2 m_h} \lvert \bar{\mathscr{M}} \rvert ^2 d \varphi\\
    d\varphi &= \dfrac{1}{16 \pi^2} \dfrac{\lvert p \rvert}{m_h} d \Omega
\end{align*}
Calcoliamo quindi l'elemento di matrice invariante:
\begin{align*}
    \mathscr{M} &= \dfrac{i}{2} g^2 v \, \eta_{\mu\nu}
    \varepsilon^{\mu}_{\lambda}(p_1) \varepsilon^{\nu
    *}_{\lambda^\prime}(p_2)\\
    \overline{\lvert \mathscr{M} \rvert}^2
    &= \sum_{\lambda \lambda^\prime}^{}
    \dfrac{g^4 v^2}{4}
    \varepsilon^{\mu}_{\lambda}(p_1) \varepsilon^{\nu *}_{\lambda}(p_1)     
    \varepsilon_{\mu\lambda^\prime}(p_2) \varepsilon^{*}_{\nu\lambda^\prime}(p_2)\\
    &= 
    \dfrac{g^4 v^2}{4}
    \left( \eta^{\mu\nu} - \dfrac{p_1^\mu p_1^\nu}{m^2} \right)
    \left( \eta_{\mu\nu} - \dfrac{p_{1\mu} p_{2\nu}}{m^2} \right)\\
    &= 
    \dfrac{g^4 v^2}{4}
    \left[ 4 - \dfrac{p_1^2}{m^2} - \dfrac{p_2^2}{m^2} +
    \dfrac{(p_{1\mu}p^{\mu}_2)^2}{m^4}\right] \qquad [p^2 = m^2]\\
    &= 
    \dfrac{g^4 v^2}{4}
    \left[ 4 - 2 +
    \dfrac{(p_1\cdot p_2)^2}{m^4}\right]\\
\end{align*}
Nel sistema di riferimento del centro di massa, $p_1 \cdot p_2 = m^2 + 2
\bar p ^2$. Ricordiamo anche che $m^2 = g^2 v^2 / 4$. Definiamo infine
$x = 4m^2 / m_h^2$, in modo da poter scrivere $|\bar p| = m_h\sqrt{1 - x}/2$.
\begin{align*}
    \overline{\lvert \mathscr{M} \rvert}^2
    &= 
    \dfrac{g^4 v^2}{4}
    \left[ 2 + \dfrac{(m^2 + \bar p^2)^2}{m^4} \right]\\
    &= \dfrac{m_h^4}{v^2} \left( 1 - x + \dfrac{3}{4}x^2 \right)\\
    d \Gamma
    &= 
    \dfrac{1}{2m_h} \dfrac{m_h^4}{v^2}\left( 1 - x + \dfrac{3}{4}x^2 \right)
    \dfrac{1}{16 \pi^2} \dfrac{1}{2}\sqrt{1 - x} d\Omega\\
    \Gamma
    &= 
    \dfrac{1}{16\pi}\dfrac{m_h^3}{v^2} \left( 1 - x + \dfrac{3}{4}x^2 \right)\sqrt{1 - x}
\end{align*}
\end{exercise}
\begin{exercise}
    Dimostrare che per il decadimento $h \rightarrow \zboson\zboson$ vale una formula
    analoga, sostituendo la massa della $\zboson$ alla massa del $\wboson$.
    \begin{equation*}
        \Gamma(h \rightarrow \zboson \zboson) = \dfrac{1}{2} \Gamma(h
        \rightarrow \wboson \wboson) \quad (\text{con la sostituzione } m_{\zboson} \leftrightarrow
        m_{\wboson})
    \end{equation*}
\end{exercise}
\subsection{Parte fermionica}
Cominciamo con una sola generazione di fermioni: $u$, $d$, $\nu$, $e$.
Scriviamo anche le loro propriet\`a di trasformazione sotto il gruppo di
gauge $SU(2)_L \times U(1)_Y$.
\begin{table}[h!]
    \centering
    \begin{tabular}{r cc}
        \toprule
        fermione & $SU(2)_L$ & $U(1)_Y$ \\
        \midrule
        $\ell_L = 
        \begin{pmatrix}
            \nu_L \\ e_L
        \end{pmatrix}$
        & $2$ & $-\nicefrac{1}{2}$\\[.3cm]
        $e_R$ & $1$ & $-1$ \\[.3cm]
        $\nu_R$ & \multicolumn{2}{c}{non esiste} \\[.3cm]  
        $q_L = 
        \begin{pmatrix}
            u_L \\ d_L
        \end{pmatrix}$
        & $2$ & $+\nicefrac{1}{6}$\\[.3cm]  
        $u_R$ & $1$ &  $+\nicefrac{2}{3}$\\ [.3cm]  
        $d_R$ & $1$ &  $-\nicefrac{1}{3}$\\
        \bottomrule
    \end{tabular}
    \caption{propriet\`a di trasformazione dei campi fermionici, per il
    modello standard con una sola generazione.}
    \label{tab:fermioni}
\end{table}
Al di l\`a dei risultati sperimentali che le confermano, queste propriet\`a
di trasformazione non si possono scegliere con arbitrariet\`a totale.
Infatti, la teoria quantistica deve essere priva di anomalie. Si parla di
\emph{anomalia} quando la teoria quantistica non conserva delle correnti che
sono invece conservati nel modello classico corrispondente. Altri limiti
possono venire dalle teorie di grande unificazione.

Per leggere la fisica del settore fermionico, scriviamo le derivate
covarianti per tutti i nuovi campi.

\begin{align*}
    \Dimu \ell_L &= 
    (\dimu + i g W_{a\mu}\dfrac{\sigma^a}{2} - i g^{\prime} B_\mu) \ell_L\\
    \Dimu e_R &= (\dimu - i g^\prime B_\mu) e_R\\
    \Dimu q_L &= 
    (\dimu + i g W_{a\mu}\dfrac{\sigma^a}{2} + \dfrac{i}{6} g^{\prime}
    B_\mu) q_L\\
    \Dimu u_R &= (\dimu + \dfrac{2i}{3} g^\prime B_\mu) u_R\\
    \Dimu d_R &= (\dimu - \dfrac{i}{3} g^\prime B_\mu) d_R\\
\end{align*}
La lagrangiana contiene dunque circa cinquanta termini di interazione tra i
vari campi. La potenza del principio di gauge \`e tale per cui tutte queste
interazioni sono controllate solo da due parametri: $g$ e $g^\prime$.
Riscriviamo la lagrangiana in termini dei campi $W^{\pm\mu}$, $Z^\mu$ e
$A^\mu$.
\begin{equation*}
    i \bar{\ell}_L \gamma^\mu \Dimu \ell_L + 
    i \bar{e}_R \gamma^\mu \Dimu e_R =
    \text{termini cinetici} +
    i \bar{\ell}_L \gamma^\mu \left( i g W_{a\mu} \dfrac{\sigma^a}{2} -
    \dfrac{i}{2} g^{\prime} B_\mu \right) \ell_L + 
    i \bar{e}_R \gamma^\mu (-i g^{\prime} B_\mu) e_R
\end{equation*}
Che possiamo dividere in una parte di \emph{corrente carica} e una parte di
\emph{corrente neutra}.
\begin{align*}
    \text{cc} &=
    -\bar{\ell}_L \gamma^\mu \dfrac{g}{2}
    \begin{pmatrix}
        0 & W_{1\mu} - i W_{2\mu}\\
        W_{1\mu} + i W_{2\mu} & 0
    \end{pmatrix}
    \ell_L\\
    &= 
    -\dfrac{g}{2}\bar{\ell}_L \gamma^\mu 
    \begin{pmatrix}
        & \sqrt{2} W^+_\mu\\
        \sqrt{2} W^-_\mu & 
    \end{pmatrix}
    \ell_L\\
    &=
    -\dfrac{g}{\sqrt{2}} \bar{\nu}_L \gamma^\mu W^+_\mu e_L
    -\dfrac{g}{\sqrt{2}} \bar{e}_L \gamma^\mu W^-_\mu  \nu_L\\
    \\
    \text{nc}
    &= 
    -\bar{\ell}_L \gamma^\mu (i g W_{3\mu}T^3 + g^{\prime} B_\mu Y) \ell_L
    -\bar{e}_R \gamma^\mu ig^{\prime} B_\mu Y e_R
\end{align*}
Riscriviamo quest'ultimo termine in termini dei campi $Z$ e $A$, con la
relazione inversa della~\eqref{eq:rotazione_weinberg}, abbreviando
$c = \cos \theta$ e $s = \sin \theta$ e sopprimendo gli indici di
Lorentz.
\begin{align*}
    W_3 &= cZ + sA\\
    B &= -sZ + cA\\
    \\
    i g W_{3\mu}T^3 + g^{\prime} B_\mu Y &= 
    g (cZ + sA) T^3 +
    g^{\prime}(-sZ + cA) Y\\
    &= 
    A (gsT^3 + g^{\prime}c Y) +
    Z (gcT^3 - g^{\prime}s Y)\\
    &= 
    eAQ + 
    Z(\sqrt{g^2 + g^{\prime 2}} T^3 - g^{\prime}s Q)\\
    &= 
    eAQ + 
    \sqrt{g^2 + g^{\prime 2}} Z(T^3 - s^2 Q)\\
\end{align*}
Il risultato finale \`e la lagrangiana
\begin{equation*}
    \lagrfer =
    i \bar{f}\gamma^\mu\dimu f -
    \dfrac{g}{\sqrt{2}}W^+_\mu J^{\mu}_- + \text{h.c.} -
    e A_\mu J^{\mu}_{\text{em}}
    - \dfrac{g}{\cos \theta} Z_\mu J^\mu_{\text{nc}}
\end{equation*}
Con le correnti
\begin{align*}
    J^{\mu}_- &= \bar{\nu}_L \gamma^\mu e_L + \bar{u}_L \gamma^\mu d_L \\
    J^{\mu}_{\text{em}} &= -\bar{e}_L \gamma^\mu e_L
    + \dfrac{2}{3}\bar{u}_L \gamma^\mu u_L
    - \dfrac{1}{3}\bar{d}_L \gamma^\mu d_L
    \\
    J^{\mu}_{\text{nc}} &= J^\mu_{3L} - \sin^2 \theta J_{\text{em}}^\mu\\
    J^\mu_{3L} &= \bar{\ell}_L \gamma^\mu \dfrac{\sigma^3}{2}\ell_L +
    \bar{q}_L \gamma^\mu \dfrac{\sigma^3}{2}q_L 
\end{align*}
\subsection{La lagrangiana efficace di Fermi}
Possiamo raccordare i risultati della teoria elettrodebole con la
lagrangiana efficace di Fermi, con l'interazione a quattro fermioni. Bastano
tre semplici passaggi.
Partiamo da una teoria pi\`u semplice come esempio, con un campo bosonico
$\varphi$ di massa $M$ e un solo campo fermionico $\psi$ di massa $m \ll M$.
La lagrangiana \`e quindi:
\begin{equation*}
    \lagr = \dfrac{1}{2}\partial_\mu \varphi \partial^\mu \varphi -
    \dfrac{1}{2}M^2 \varphi^2 + i \bar{\psi}\slashed{\partial} \psi
    - m \bar{\psi}\psi - g \varphi \bar{\psi} \psi.
\end{equation*}
\begin{enumerate}
    \item Scriviamo le equazioni del moto per il campo pesante, che \`e
        quello da eliminare nella teoria.
        \begin{equation*}
            (\square + M^2) \varphi = -g \bar{\psi} \psi.
        \end{equation*}
    \item Risolviamo le equazioni nel limite di $E \ll M$, trascurando il
        termine di energia cinetica, che \`e il $\square$.
        \begin{equation*}
            \varphi = -\dfrac{g}{M^2}\bar{\psi}\psi.
        \end{equation*}
    \item Sostituiamo nella lagrangiana di partenza:
        \begin{align*}
            \lagr_{\text{eff}} &= -\dfrac{1}{2}
            \dfrac{g^2}{M^2}
            \bar{\psi}\psi \bar{\psi}\psi
            +
            \dfrac{g^2}{M^2}
            \bar{\psi}\psi \bar{\psi}\psi
            + \text{termini cinetici}\\
            &= \dfrac{1}{2}
            \dfrac{g^2}{M^2}
            \bar{\psi}\psi \bar{\psi}\psi
            + \bar{\psi}(i\slashed{\partial} - m) \psi
        \end{align*}
        trovando quindi il termine di interazione a quattro fermioni.
\end{enumerate}
Applichiamo la procedura alla teoria elettrodebole con lagrangiana:
\begin{equation*}
    \lagr_{\text{SM}} = \text{term. cin.} + m_W^2 W_\mu^+
    W^{-\mu} + \dfrac{1}{2}m_Z^2 Z^\mu Z_\mu
    - \dfrac{g}{\sqrt{2}}(W^{+}_\mu J^{-\mu} + \text{h.c.})
    - e A_\mu J^\mu_{\text{EM}}
    - \dfrac{g}{\cos^2 \theta} Z_\mu J_Z^\mu
    + \cdots
\end{equation*}
\begin{enumerate}
    \item le equazioni del moto, trascurando le energie cinetiche, d\`anno:
        \begin{align*}
            m_W^2 W^{-\mu} &= \dfrac{g}{\sqrt{2}}J^{- \mu}
             \qquad W^{\pm \mu} = \dfrac{g}{\sqrt{2}m_W^2} J^{\pm \mu}\\
            m_Z^2 Z^{\mu} &= \dfrac{g}{\cos^2 \theta}J_Z^{\mu}
             \qquad Z^{\mu} = \dfrac{g}{m_Z^2 \cos^2\theta} J^{\mu}_Z
             =\dfrac{g}{m_W^2} J^{\mu}_Z  
        \end{align*}
    \item otteniamo la lagrangiana efficace, che dipende solo dai campi
        fermionici e dei fotoni:
        \begin{equation*}
            \lagr_{\text{eff}} =
            -\dfrac{1}{4}F_{\mu\nu}F^{\mu\nu}
            - e A_\mu J^\mu_{\text{EM}}
            - \dfrac{g^2}{2m_W^2} J^{+\mu}J_\mu^-
            - \dfrac{g^2}{2m_Z^2 \cos^2\theta} J_Z^\mu J_{Z\mu}
            + \cdots
        \end{equation*}
        dove si introduce di solito la costante di Fermi.
        \begin{definition}
            [costante di Fermi]
            \begin{equation*}
                \dfrac{G_F}{\sqrt{2}} \mathop: = \dfrac{g^2}{8m_W^2} =
                \dfrac{1}{2 v^2}
            \end{equation*}
        \end{definition}
\end{enumerate}
Possiamo per esempio applicare questa teoria al decadimento del muone,
determinato dal pezzo di lagrangiana che coinvolge la corrente carica:
\begin{equation*}
    \lagr = -\dfrac{G_F}{\sqrt{2}}\bar{e} \gamma^\alpha(1 - \gamma^5) \nu_e
   \bar{\nu}_\mu\gamma_\alpha(1 - \gamma^5) \mu.
\end{equation*}
\subsection{La vita media del muone}
Il calcolo della vita media del muone \`e importante perch\'e permette di
ricavare con precisione il valore di $G_F$ che, come abbiamo visto, \`e uno
dei quattro parametri fondamentali del modello standard.
\section{Test di precisione del modello standard}
Sono stati misurati con grande precisione sperimentale i parametri
fondamentali del modello standard:
\begin{align*}
    \Gamma(\zboson \rightarrow e^+ e^-) &= 83.925(86) MeV\\
    \sin^2\theta_{\wboson\text{,lep}} &= 0.23153(16)\\
    g_{v}^{e} &=  -0.03783(41)
\end{align*}
Un tale livello di accuratezza sperimentale, dell'ordine del per mille
richiede di correggere le previsioni teoriche con grafici radiativi. Infatti
$\alpha/\pi$ \`e dell'ordine del per cento.
Ci\`o \`e molto interessante, perch\'e nei \emph{loop} ci sono anche le
particelle non ancora rivelate direttamente. Dai test di precisione del
modello standard \`e quindi possibile ricavare informazioni sulla massa del
bosone di Higgs.
\subsection{Evoluzione di $\alpha$}
\begin{figure}[h]
    \begin{center}
        \begin{fmffile}{propagatorefotone}
\begin{fmfgraph*}(100,100)
\fmfleft{i}
\fmfright{o}
\fmf{boson,tension=1.2}{i,v1}
\fmf{boson,tension=1.2}{v2,o}
\fmf{plain,left}{v1,v2,v1}
\end{fmfgraph*}
\end{fmffile}

    \end{center}
    \caption{correzioni al propagatore del fotone.}
    \label{fig:propagatorefotone}
\end{figure}
Bisogna invertire la parte quadratica della lagrangiana che dipende dal
campo del fotone.
\begin{align*}
    \lagr_2 &= -\dfrac{1}{4}F_{\mu\nu}F^{\mu\nu} - \dfrac{\xi}{2}(\dimu
    A^\mu)^2\\
    &= \dfrac{1}{2}A_\mu(\box g^{\mu\nu} - \partial^\mu \partial^\nu) A_\nu
    +\dfrac{\xi}{2}A_\mu\partial^\mu \partial^\nu A_\nu\\                                      
    &= \text{(Fourier) } 
    \dfrac{1}{2}A_\mu(\box g^{\mu\nu} - q^\mu q^\nu) A_\nu
    +\dfrac{\xi}{2}A_\mu q^\mu q^\nu A_\nu
\end{align*}
Il propagatore vale dunque
\begin{align*}
    \left[ -iq^2(g^{\mu\nu} - \dfrac{q^\mu q^\nu}{q^2}) -i \xi q^2
    \dfrac{q^\mu q^\nu}{q^2}\right]^{-1} = 
    -\dfrac{i}{q^2}(g^{\mu\nu} - \dfrac{q^\mu q^\nu}{q^2}) - \dfrac{i}{\xi
    q^2} \dfrac{q^\mu q^\nu}{q^2}
\end{align*}
Con la correzione di aggiunge a questo termine
\begin{equation*}
    -i \Pi^{\mu\nu}_{\gamma\gamma}(q) \text{ con } 
    \Pi^{\mu\nu}_{\gamma\gamma}(q) = F_{\gamma\gamma}(q^2)q^2(g^{\mu\nu} -
    \dfrac{q^\mu q^\nu}{q^2})
\end{equation*}
Dove $F_{\gamma\gamma}$ \`e un termine divergente. Occorre dunque
rinormalizzare il campo $A_\mu \rightarrow \sqrt{Z_3}A_\mu$.
\end{document}
